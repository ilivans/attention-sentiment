\chapter{Заключение}

Таким образом, исследована применимость модели на основе двунаправленной рекуррентной нейронной сети с механизмом внимания в задаче классификации тональности русскоязычных текстов. Проведён подбор гиперпараметров и обучены модели на нескольких наборах данных. Проведено сравнение данной модели с её ранее изученными аналогами. Выявлено преимущество использования предложенной архитектуры в случае достаточно длинных документов (порядка 300 слов), а именно, для рецензий на фильмы с ресурса \url{www.kinopoisk.ru}.

В рамках данной работы реализован алгоритм двунаправленной рекуррентной нейронной сети с механизмом внимания. Код отлажен и выложен в открытый доступ \url{www.github.com/ilivans/tf-rnn-attention}. Также доступен весь код, использованный в работе, значения гиперпараметров моделей и сам отчёт \url{www.github.com/ilivans/attention-sentiment}.

Для дальнейшего исследования предлагается изучить применимость более сложных и глубоких архитектур для анализа тональности русскоязычных текстов, а также применимость данной модели в качестве модуля для нейронной сети, генерирующей сообщения с заданной тональностью.
